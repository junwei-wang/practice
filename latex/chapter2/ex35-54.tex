%%%%%%%%%%%%%%%%%%%%%%%%%%%%%%%%%%%%%%%%%%%%%%%%%%%%%%
% Filename:      ex35-54.tex
% Author:        Junwei Wang(wakemecn@gmail.com)
% Last Modified: 2012-03-11 20:43
% Description:
%%%%%%%%%%%%%%%%%%%%%%%%%%%%%%%%%%%%%%%%%%%%%%%%%%%%%%
\documentclass{book}
\usepackage[space]{ctex}
\usepackage{graphpap}
\usepackage{fancyvrb}
\begin{document}
\begin{center}
{\large 高等数学讲义}\\[4mm]
下册\\[1mm]
wakemecn@gmail.com\\
\end{center}

{\centering 
{\large 高等数学讲义}\\[4mm]
下册\\[1mm]
wakemecn@gmail.com\\
}

\begin{flushleft}
f:你好\\
E:你好\\
D:你好\\
V:你好\\
\end{flushleft}

\begin{flushright}
QQ:552026291\\
Tel:15154101098\\
Email:wakemecn@gmail.com\\
Add:No.1500, Shunhua Road, Jinan City\\
Shandong Prov. China 250101 
\end{flushright}

\begin{picture}(200,170)(-20,-18)
\graphpaper(0,0)(150,150)
\thicklines\color{blue}
\put(17,22){\line(1,0){100}}
\put(117,22){\line(0,1){100}}
\put(17,22){\line(1,1){100}}\color{red}
\put(74,11){1}
\put(122,71){1}
\put(63,90){$\sqrt{2}$}
\end{picture}

\begin{quotation}
黑洞是恒星的一种残骸,它是引力收缩的极点,极端到近乎荒唐。\par
%\raggedleft ---约翰\cdot卢米涅
\end{quotation}
黑洞一次是在1968年由美国天体物理学家慧勒首先提出来的。
\\\\\\\\\\
A human book designer tries to find out waht the author had in mind while %
writing the manuscript. He decides on chapter headings, citations, examples, %
formulat,etc.\\
\begin{sloppypar}
A human book designer tries to find out waht the author had in mind while %
writing the manuscript. He decides on chapter headings, citations, examples, %
formulat,etc.\\
\end{sloppypar}

\begin{verbatim}
#include <stdio.h>
static int test(char * str) {
	return str;
}
\end{verbatim}
\begin{verbatim*}
#include <stdio.h>
static int test(char * str) {
	return str;
}
\end{verbatim*}
\makeatletter
\renewcommand{\verbatim@font}{\sffamily\slshape\small}
\makeatother
\begin{verbatim}
#include <stdio.h>
static int test(char * str) {
	return str;
}
\end{verbatim}

\begin{Verbatim}
#include <stdio.h>
static int test(char * str) {
	return str;
}
\end{Verbatim}


\begin{Verbatim}[numbers=left,commandchars=\\\{\}]
#include <stdio.h>
static int test(char * str) \{
	str = str++;
	return \label{verbtext};
\}
\end{Verbatim}
上面例子中第\ref{verbtext} 行语句是返回主程序命令。

\begin{Verbatim}[numbers=left,commandchars=+\[\]]
#include <stdio.h>
static int test(char * str) {
	str = str--;
	return +label[verbtext];
}
\end{Verbatim}
上面例子中第\ref{verbtext} 行语句是返回主程序命令。

\VerbatimFootnotes
脚注命令\footnote{\verb"\footnote"}中可以使用抄录命令。

\begin{verse}
The Red Army fears not the trials of the Long March,\\
Holding light ten thousand crags and torrets.\\...\\
\end{verse}

\newenvironment{Proof}{{\noindent \em Proof}\par}{\hfill $\diamondsuit$\par}
\begin{Proof}
This words proof that words;
\end{Proof}

\newenvironment{Theorem}[1][]{\par\noindent{\heiti 定理}#1\quad}{\par}
\begin{Theorem}[(必要条件)]
设可微分的函数...
\end{Theorem}

\end{document}
