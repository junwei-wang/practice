%%%%%%%%%%%%%%%%%%%%%%%%%%%%%%%%%%%%%%%%%%%%%%%%%%%%%%
% Filename:      ex16-33.tex
% Author:        Junwei Wang(wakemecn@gmail.com)
% Last Modified: 2012-03-11 20:52
% Description:
%%%%%%%%%%%%%%%%%%%%%%%%%%%%%%%%%%%%%%%%%%%%%%%%%%%%%%
\documentclass{book}
\usepackage[space]{ctex}
\usepackage{graphicx}
\usepackage{fancybox}
\begin{document}
\parindent=0pt
\makebox[60mm][s]{电子计算机}\\
\makebox[60mm][s]{computer}\\
\makebox[60mm][s]{computer test}\\
\makebox[60mm][s]{c o m p u t e r}\\

For emphasis, you may wish to \raisebox{1,5ex}{raise} or %
\raisebox{-1.0ex}{lower} certain text insides your documents.

For emphasis, you may wish to \raisebox{2,5ex}{raise} or %
\raisebox{-1.0ex}{lower} certain text insides your documents.

\noindent \makebox[0pt][r]{fbox{注意} \qquad 如果重新安装操作%
则系统盘中的所有数据将被删除!}

\noindent \makebox[0pt][r]{\rotatebox{90}{\makebox[0pt][r]{零宽度盒子}%
\quad}}\quad 尽管对象仍然被正确地被排版,但系\\统认为他的宽度为零,与%
前面的或后\\面的文本排版无关,从而造成相互重叠。\\通常这种现象显然不正%
常,但有时却可\\以利用宽度盒子的这一特点。制作出多\\种页面特效。

$$a^{2}+b^{2}=c^{2}$$
$$a^{2}+b^{2}=c^{2}\makebox{(勾股定理)}$$
$$a^{2}+b^{2}=c^{2}\makebox[0pt][l]{(勾股定理)}$$

\newlength{\mylen}
\settowidth{\mylen}{勾股定理}
\makebox[0pt][l]{\color{blue}\rule[-0.9ex]{\mylen}{1pt}}勾股定理:%
直角三角形两直角边的平方和等于斜边的平方。

\noindent 分 类 号:U491 \hfill
\newlength{\mylength}
\settowidth{\mylength}{学\qquad 号:200800300237}
\begin{minipage}[t]{\mylength}
单位代码:10000\\
学\qquad 号:200800300237\\
密\qquad 级:公开\\
\end{minipage}

\noindent 分 类 号:U491 \hfill
\begin{minipage}[b]{\mylength}
单位代码:10000\\
学\qquad 号:200800300237\\
密\qquad 级:公开\\
\end{minipage}


\noindent 楞次定律:\parbox[t]{43mm}{\sl 磁通变化产生的电流的方向是%
它能阻止磁通变化的方向上。}


\parbox{37mm}{Did you know?\par \rule[4mm]{37mm}{1.5pt}}

\framebox{\rule{9mm}{1pt}\rule{1pt}{5mm}}
\framebox{\rule{9mm}{0pt}\rule{0pt}{5mm}}

\noindent \fbox{\parbox{57mm}{\parindent=4pt 请在整个关机过程结束后\\再关闭计算机}}

\framebox[2\width]{computer}

\newsavebox{\mysquare}
\sbox{\mysquare}{\fboxrule=1pt \framebox{\rule{7mm}{0pt}\rule{0pt}{7mm}}}

\usebox{\mysquare}
\usebox{\mysquare}

\newsavebox{\mybox}
\newenvironment{FramePage}[2][c]%
{\begin{lrbox}{\mybox}\begin{minipage}[#1]{#2}}%
{\end{minipage}\end{lrbox}\shadowsize=2pt \shadowbox{\usebox{\mybox}}}
\begin{FramePage}{58mm}
在小页环境中可以使用抄录命令和抄录环境。这是千真万确的。
\end{FramePage}

\end{document}
